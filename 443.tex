\documentclass[10pt,letterpaper,landscape]{report}

\usepackage[utf8]{inputenc}
\usepackage[english]{babel}
\usepackage[T1]{fontenc}
\usepackage{lmodern}


\usepackage{amsmath}
\usepackage{amsfonts}
\usepackage{amssymb}

\usepackage{enumitem}
\setlist{nosep}


\usepackage{graphicx}

\usepackage[left=-0.5cm,right=-0.5cm,top=-0.3cm,bottom=-0.3cm]{geometry}


\author{Adrien Cosson}
\title{Fiches de révision}


\newcommand{\boxheight}{21.59cm}
\newcommand{\boxwidth}{8.85cm}



\begin{document}
\begin{small}
\fbox{
\begin{minipage}[t][\boxheight][c]{\boxwidth}



	\textbf{Basic Terminology}
	\begin{itemize}
		\item Confidentiality: Communication is possible without the attacker learning the contents of the conversation
		\item Integrity: Verifying that the message has been received intact
		\item Authentication: Making sure that the participants in the conversation are whoever they are
		\item Anonymous communication: Allow the sender to communicate anonymously with a recipient
	\end{itemize}

	\textbf{Threats}\\
	Threat: means by which an attacker can put a system at risk\\
	Threat model: collection of threats deemed important in a certain environment (collection of attacker abilities)
	
	\textbf{Vulnerability}\\
	Systematic artefact that exposes data, user or system to a threat
	
	\textbf{Adversary}\\
	Entity trying to circumvent a security infrastructure\\
	$\rightarrow$ an \textit{insider} is an adversary that has legitimate access
	
	\textbf{Attack}\\
	When an attacker tries to exploit a vulnerability
	\begin{itemize}
		\item Passive
		\item Active
		\item DoS
	\end{itemize}
	$\rightarrow$ a \textit{compromise} is when an attack is successful
	
	\textbf{Trust}\\
	Degree to which we expect an entity to behave\\
	$\rightarrow$ a \textit{trust model} describes who is trusted in a particular environment
	
	\textbf{Principles of security}
	\begin{enumerate}
		\item Open Design: no "security through obscurity"
		\item Least Privilege: just grant the privileges required to perform a function
		\item Fail-Safe Defaults: if a permission is not explicitly granted, is should be denied (ie. if you fail a permission check, stop
		\item Economy of Mechanism : keep it simple to keep it safe
		\item Complete Mediation : every access to any resource is checked for authorization
		\item Separation of Privilege : a system should use multiple authentications
		\item Least Common Mechanism : minimize common mechanisms between users
		\item Psychological Acceptability : system needs to be ergonomic
	\end{enumerate}

	\textbf{IoT}\\
	Devices that integrate physical processes with digital connectivity
	
	\textbf{Crypto Terminology}\\
	\textit{Cryptosystem}: method of disguising plaintext messages that only select parties can decipher\\
	\textit{Cryptography}: the art/science of developing cryptosystem\\
	\textit{Cryptanalysis}: the art/science of breaking cryptosystem\\
	\textit{Cryptology}: The combination of cryptography and cryptanalysis
	
	\textbf{Shared/Symmetric Key Cryptography}\\
	$D(E(p,k),k) = C$ where $k$ is the shared key
	
	\textbf{One-Time Pad (OTP)}\\
	To obtain ciphertext, XOR the plaintext with the secret key (the one-time pad), where $len(plaintext) == len(OPT)$\\
	$\mathbb{P}[M=m\vert C=c] = \mathbb{P}[M=m] \rightarrow$ you learn nothing more by having the ciphertext
	



\end{minipage}
}\fbox{
\begin{minipage}[t][\boxheight][c]{\boxwidth}


	\textbf{Two kinds of confidentiality}
	\begin{itemize}
		\item Unconditional/probabilistic security: offer provable guarantee, irrespective of computational abilities\\
		$\rightarrow len(key) \approx len(message)$
		\item Conditional/computational security: cryptosystem is secure under the assumption of a computationally bounded adversary\\
		$\rightarrow len(key) \ll len(message) $
	\end{itemize}

	\textbf{Cipher Attack Classes}
	\begin{itemize}
		\item Ciphertext-only: adv. only has the ciphertext
		\item Known-plaintext: adv. has pairs of plaintext/ciphertext
		\item Chosen-plaintext: adv. can obtain the ciphertext of any plaintext
		\item Chosen-ciphertext: adv. can obtain the plaintext of any ciphertext (except the target one)
	\end{itemize}

	\textbf{Kerckhoff's Principles}
	\begin{enumerate}
		\item Assume Eve knows the cipher algorithm
		\item Security should only depend on the key
		\item If the key is compromised, it should be able to be changed
	\end{enumerate}

	\textbf{Key Reuse in the OTP encryption}\\
	XOR-ing two different ciphertexts results in the XOR-ing of the two plaintexts: 
	$C_1 \oplus C_2 = P_1 \oplus k\ \oplus P_2 \oplus k = P_1 \oplus P_2$\\
	This can help obtain a specific plaintext, that can then be used to obtain the key: $ P \oplus C = P \oplus P \oplus k = k$
	
	\textbf{Hash Function Requirement:}
	\begin{itemize}
		\item Fixed size digest
		\item Irreversibility
		\item Few collisions
	\end{itemize}

	\textbf{Key management}\\
	\textit{Key Distribution}: process where we assign and transfer keys to a participant\\
	\textit{Key Agreement}: process where multiple parties negotiate a key
	
	\textbf{Diffie-Hellman key exchange}\\
	Public information: $p$ is a prime, $g$ is a primitive root of $p$ ($g < p$)\\
	Participants pick $x_1$ and $x_2$ ($x < p-1$)\\
	$y_1 = g^{x_1} \mod p$\\
	$z = y_2 ^ {x_1} \mod p$
	
	\textbf{Diffie-Hellman vulnerability}\\
	Man-in-the-middle attack, because DH does not implement any authentication
	
	\textbf{Public/asymmetric cryptography}\\
	$D(E(p, k_e), k_d) = p$\\
	$E(D(p, k_d), k_e) = p$
	
	\textbf{RSA keygen}
	\begin{enumerate}
		\item Pick two large primes $p$ and $q$
		\item Calulate $n = p*q$
		\item Pick $e$ such as it is coprime with $\Phi(n) = (q-1)*(p-1)$
		\item Calculate $d$ such as	TO COMPLETE
	\end{enumerate}
	Public key: ${e, n}$\\
	Private key: ${d, n}$\\
	$E(k_e, p) = p^d \mod n$\\
	$D(k_d, p) = c^e \mod n$\\
	Hard to break because it requires to factor $n$ as $p*q$

\end{minipage}
}\fbox{
\begin{minipage}[t][\boxheight][c]{\boxwidth}


	\textbf{Digital Signature Properties}
	\begin{itemize}
		\item Not forgeable
		\item Authenticity check
		\item Not alterable
		\item Non-repudiation
	\end{itemize}

	\textbf{Authentication Vocabulary}\\
	\textit{Credential}: proof of identitfy\\
	\textit{Evaluation}: Processes used to asses the link between an identity and a credential
	
	\textbf{Certificate}
	\begin{itemize}
		\item Associates a user and a private key
		\item Contains pubkey information
		\item Has a validity period
		\item Is signed by a Certificate Authority (CA)
		\item Identity may be vetted by a Registration Authority (RA)
	\end{itemize}

	\textbf{Root CA certificate}\\
	Certificate baked into the system (browser, OS) by default, that is the original root of trust
	
	\textbf{Certificate Management terminology}\\
	\textit{Issuer}: agent that issues/signs a certificate\\
	\textit{Subject}: agent whose key is about\\
	\textit{Verifier}: agent who wants to check someone's else key\\
	\textit{Target}: agent whose key is being searched for\\
	\textit{Principal}: anything that has a public key\\
	\textit{Trust anchor}: anything trusted to sign certificate
	
	\textbf{Public Key Infrastructure (PKI)}\\
	Rooted tree of CA, where any CA can issue certificates for its children $\rightarrow$ generates a cascading trust chain\\
	$\rightarrow$ Issue: any CA can generate malicious children CA
	
	\textbf{PKI Issues}
	\begin{enumerate}
		\item Why/how can we trust a CA? No legal liability to run a bad CA, only incentive is the risk of going bankrupt
		\item Who is using my key? Private key is insecure
		\item How secure is the verifier? The verifier is vulnerable, the root pubkey has to be kept safe
		\item How to identify the owner of a private key? No publicly known identifier for people
		\item Is the CA an authority? CA can claim authenticity on elements it does not have any control
		\item Is the user part of the design? Too many things are hidden to the user
		\item Was it one CA or CA+RA? Separation of CA and RA can allow forgery
		\item How was the user authenticated in the first place?
		\item How secure are the certificate practises? People don't use them correctly
		\item Why are we using PKI? Certificates are not the perfect solution to authenticate a user, it just identify a key
	\end{enumerate}

	\textbf{Identity base Cryptography (IBE)}\\
	Use an arbitrary string as a public key (i.e. your email address). A trusted third party generates a private key for everyone based on their arbitrary string. The TTP is then not used anymore, as the keypair is in possession of the used.



\end{minipage}
}

\fbox{
\begin{minipage}[t][\boxheight][c]{\boxwidth}
		
		
		
		\textbf{IBE issues}
		\begin{itemize}
			\item No universal TTP
			\item Still ambiguity as to who someone is (there are multiple John Smith in the world)
			\item Revocation is not fixed, even worse
		\end{itemize}
	
		\textbf{Biometrics issues}
		\begin{itemize}
			\item No revocation possible
			\item "Fuzzy" credential
			\item Privacy issues
		\end{itemize}
	
		\textbf{Challenge/Response Authentication} \\
		Server sends a nonce, client responds with: \\ $response = H(H(name:realm:password):nonce:H(request))$
		
		\textbf{Protocol notation}\\
		$d$: data, payload\\
		$pw^A$: password for $A$\\
		$k_{AB}$: symmetric key between $A$ and $B$\\
		$K_A^+$, $K_A^-$: public/private keypair of $A$\\
		$E(k, d)$: $d$ encrypted with $k$\\
		$D(k, d)$: $d$ encrypted with $k$\\
		$Sig(K_A^-, d$ is the signature of $d$ with by Alice\\
		$H(d)$ is the hash of data $d$
		
		\textbf{Authentication Handshake} \\
		Authenticate each other, then establish session keys
		
		\textbf{Reflection attack}\\
		Using a handshake session as the oracle for another handshake session\\
		$\rightarrow$ use different keys for A-B and B-A\\
		$\rightarrow$ don't reuse challenges \\
		$\rightarrow$ the initiator should be the first one to authenticate
		
		\textbf{Two-Way Public Key Based Authentication}\\
		Establish a session key by using Diffie-Hellman with signed partial keys exchanged
		
		\textbf{Mediated Authentication}\\
		Have a Key Distribution Center (KDC) generate a ticket that will serve as a session key with your target
		
		\textbf{Kerberos Ticket}\\
		A Kerberos ticket has the properties: 
		\begin{itemize}
		    \item Alice only can open it
		    \item Contains a session key $K^{AB}$
		    \item Contains a token that can only be opened by Bob, that contains: 
		    \begin{itemize}
		        \item Alice's identity
		        \item The session key $K^{AB}$
		    \end{itemize}
		\end{itemize}
		
    \textbf{Security Policies Goals}
    \begin{itemize}
        \item Secrecy: don't allow reading by unauthorized subjects, control where data can be written by authorized users
        \item Integrity: don't permit dependence on lower integrity code/data
        \item Availability: the necessary function must run
    \end{itemize}
    
    \textbf{Trusted Computing Base}\\
    Infrastructure that you assume will behave correctly
		
		
		
	\end{minipage}
}\fbox{
\begin{minipage}[t][\boxheight][c]{\boxwidth}
		
		
		\textbf{Protection Domain}\\
		Domain that restricts access of external parties to our computing system's resources
		
		\textbf{Policy}\\
		Set of rules that define what you can/can't/must do
		
		\textbf{Access policy enforcement}\\
		\textit{Protection state}: defines what each subject can do\\
		\textit{Reference monitor}: enforces the protection state, meets the following guarantees: 
		\begin{itemize}
		    \item Tamper-proof
		    \item Complete Mediation
		    \item Simple enough to verify
		\end{itemize}
		
		\textbf{Protection system components}\\
		Consists of a protection state, operations to modify that state, and a reference monitor to enforce that state
		
		\textbf{Access Control System}\\
		System that determines what rights a particular subject has for a set of objects. An \textit{Access Control Policy} defines theses rights
		
		\textbf{Access Matrix}\\
		Way to represent a policy for a set of subjects and objects
		
		\textbf{Protection vs Security}\\
		\textit{Protection}: Security goals are met under trusted processes, protects against an error by a non-malicious entity\\
		\textit{Security}: Security goals are met under all processes, protects against malicious entities
		
		\textbf{Access Control Administration}
		\begin{itemize}
		    \item Discretionary: the owner defines the policy
		    \item Mandatory: the environment enforces a static policy
		\end{itemize}
		
		\textbf{Label}\\
		Signifier of some identity or characteristic in a MAC system
		
		\textbf{Access Matrix for DAS vs MAC}\\
		For DAC: 
		\begin{itemize}
		    \item Subjects: users
		    \item Objects: files, sockets, ...
		    \item Administration: user
		\end{itemize}
		For MAC: 
		\begin{itemize}
		    \item Subjects: labels
		    \item Objects: labels
		    \item Administration: external
		\end{itemize}
		
		\textbf{Role}\\
		Collection of privileges \& permissions associated with some function or affiliation
		
        \textbf{Network effect}\\
        The more people use a protection system, the more effective the system gets 
        
        \textbf{End-to-end argument}\\
        In order to optimize a network, make the center of it very dumb, and the edges smart
        
        \textbf{Network terminology}\\
        LAN: Local area network\\
        Edge: Point in a LAN through which the network connects to a larger area network
        

	\end{minipage}
}\fbox{
\begin{minipage}[t][\boxheight][c]{\boxwidth}
		
		
		\textbf{IP layers}\\
        Physical - Link (framing between both ends of the link) - Network (routing between two points) - Transport (transfer from source to destination) - Application
        
        \textbf{High-jacking traffic}\\
        A malicious router in a network advertises a bad route metric to cause congestion

		\textbf{DNSsec}\\
		PKI infrastructure over DNS, to provide signature of every DNS message by all the layers in the DNS chain
		
		\textbf{Communication security}
		\begin{itemize}
		    \item Authenticate end points
		    \item TODO
		\end{itemize}
		
		\textbf{IPsec}\\
		Transparent security suite, that operates at the IP layer. Creates a new layer between IP and TCP/UDP: AH (authentication header, provides integrity) and ESP (encapsulating security payload, provides confidentiality)
		
		\textbf{IPsec modes of operation}
		\begin{itemize}
		    \item Tranport: encrypt the payload, MAC part of the header (non-mutable fields)+encrypted payload
		    \item Tunnel: encrypt everything, MAC it and add a new header
		\end{itemize}
		
		\textbf{Firewall types}
		\begin{itemize}
		    \item IP Firewall, defines a policy based on address, port, protocol, interface, flags,\dots
		    \item Proxy Firewall, at the application level: analyses the transaction and decides if it should be forwarded
		\end{itemize}
		
		\textbf{X-listing}\\
		Blacklisting: specifying specific connectivity that is explicitly disallowed \\
		Whitelisting: specifying specific connectivity that is explicitly allowed
		
		\textbf{Stateful vs Stateless}\\
		Stateful: makes a decision based on information that happened before\\
		Stateless: only uses the current transaction information to make a decision
		
		
		\textbf{DMZ}\\
		De-militarized zone: sub-network quarantined from internal network and internet facing services
		
		\textbf{NIDS} \\
		Network Intrusion Detection System
		
		
	\end{minipage}
}

\fbox{
\begin{minipage}[t][\boxheight][c]{\boxwidth}
		
		
		
		TEST
		
		
		
		
	\end{minipage}
}\fbox{
	\begin{minipage}[t][21cm][c]{9.5cm}
		
		
		TEST
		
	\end{minipage}
}\fbox{
	\begin{minipage}[t][21cm][c]{9.5cm}
		
		
		TEST
		
		
		
	\end{minipage}
}
\end{small}
\end{document}
