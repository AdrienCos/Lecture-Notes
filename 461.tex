\documentclass[10pt,letterpaper,landscape]{report}

\usepackage[utf8]{inputenc}
\usepackage[english]{babel}
\usepackage[T1]{fontenc}
\usepackage{lmodern}


\usepackage{amsmath}
\usepackage{amsfonts}
\usepackage{amssymb}

\usepackage{enumitem}
\setlist{nosep}


\usepackage{graphicx}

\usepackage[left=-0.5cm,right=-0.5cm,top=-0.3cm,bottom=-0.3cm]{geometry}


\author{Adrien Cosson}
\title{Fiches de révision}

\newcommand{\boxheight}{21.59cm}
\newcommand{\boxwidth}{8.85cm}


\begin{document}
\begin{small}
\fbox{
\begin{minipage}[t][\boxheight][c]{\boxwidth}

    \textbf{Components of a PL}
    \begin{itemize}
        \item Syntax: what valid programs are
        \item Semantics: what a program does
        \item Design Philosophy: style of programming, programming paradigm
    \end{itemize}
    
    \textbf{Paradigms}
    \begin{itemize}
        \item Imperative: program is a sequence of commands that change the state (ex: Pascal, C)
        \item Object-oriented: program is a collection of objects and their interactions via message passing (Java, C++)
        \item Functional: program is a collection of functions as well as their interactions (Lisp, Scheme, Haskell)
        \item Logical: program is a collection of logical rules (Prolog, Datalog)
    \end{itemize}
    $\Rightarrow$ a PL can mix paradigms

    \textbf{PL Implementations}
    \begin{itemize}
           \item Interpreted language
           \item Compiled language     
    \end{itemize}

    \textbf{Compiler structure}
    \begin{itemize}
        \item Lexical analyser: split chars into lexical elements
        \item Syntax analyser: build an abstract-syntax tree from the lexical elements
        \item Semantic analyser: check correctness of AST
        \item Intermediate code generation
        \item Code optimization
        \item Code generation
    \end{itemize}

    \textbf{JIT compilation}\\
    Compile the most often used functions on the fly into machine code

    \textbf{Grammar}\\
    Formal way of specifying the syntax of a language.\\
    $\Rightarrow$ Context-Free Grammar (BNF)

    \textbf{BNF concepts for a simple arithmetic language}
    \begin{itemize}
        \item Nonterminals: grammatical categories
        \begin{itemize}
            \item <e>: expression
            \item <n>: number
            \item <d>: digit
        \end{itemize}
        \item Terminals: alphabet of the language (0, 1, \dots, 9, -, +)
        \item Production rules:
        \begin{itemize}
            \item <e> $\rightarrow$ (<n> || <e> + <e>  || <e> - <e>)
            \item <n> $\rightarrow$ (<d> || <n><d>)
            \item <d> $\rightarrow$ (0 || 1 || \dots || 9)
        \end{itemize}
        \item Start symbol: <e>
        \item Derivation: 
        \begin{enumerate}
            \item Start with the start symbol
            \item At each step, replace a non-terminal with the rhs (right-hand-side) of a rule
            \item Repeat until getting a string of terminals
            
        \end{enumerate}
    \end{itemize}


\end{minipage}
}\fbox{
\begin{minipage}[t][\boxheight][c]{\boxwidth}


	TODO
 
    
    
\end{minipage}
}\fbox{
\begin{minipage}[t][\boxheight][c]{\boxwidth}

    
    TODO
    
\end{minipage}
}

\fbox{
\begin{minipage}[t][\boxheight][c]{\boxwidth}

    TODO
    
\end{minipage}
}\fbox{
\begin{minipage}[t][\boxheight][c]{\boxwidth}

    TODO
    
\end{minipage}
}\fbox{
\begin{minipage}[t][\boxheight][c]{\boxwidth}


	TODO



\end{minipage}
}
\end{small}
\end{document}
