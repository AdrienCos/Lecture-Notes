\documentclass[10pt,letterpaper,landscape]{report}

\usepackage[utf8]{inputenc}
\usepackage[english]{babel}
\usepackage[T1]{fontenc}
\usepackage{lmodern}


\usepackage{amsmath}
\usepackage{amsfonts}
\usepackage{amssymb}

\usepackage{enumitem}
\setlist{nosep}


\usepackage{graphicx}

\usepackage[left=-0.5cm,right=-0.5cm,top=-0.3cm,bottom=-0.3cm]{geometry}


\author{Adrien Cosson}
\title{Fiches de révision}

\newcommand{\boxheight}{21.59cm}
\newcommand{\boxwidth}{8.85cm}


\begin{document}
\begin{small}
\fbox{
\begin{minipage}[t][\boxheight][c]{\boxwidth}

    \textbf{Power consumption issue}\\
    Power used by a chip is $P \approx \frac{1}{2}CV_{dd}^2Af$ \\
    Naive solution: reduce $V_{dd}$
    \begin{itemize}
        \item Leakage limits how low you can go
        \item Other solution: reduce $f$ and increase nb of cores\\
        $\implies$ needs programs to be parallelizable
    \end{itemize}
    
    \textbf{Five principles of perf.}
    \begin{itemize}
        \item Parallelism
        \item Speculation
        \item Locality
        \item Memorization
        \item Amdahl's Law
    \end{itemize}{}

    \textbf{Parallelism}\\
    $T_n$: time to compute a problem with $n$ cores \\
    Average parallelism : $P_{avg} = \frac{T_1}{T_\infty}$\\
    For a $p$ wide system:\\
    $T_p \geq \max(\frac{T_1}{p}, T_\infty)$ \\
    $P_{avg} \gg p \implies T_p \approx \frac{T_1}{p}$

    \textbf{Speculation}\\
    Make educated guesses to avoid expensive operations, if you can be right most of the time
    
    \textbf{Locality}
    \begin{itemize}
        \item Temporal: if you accessed data, you are likely to access it again soon
        \item Spatial: if you accessed data, you are likely to access its neighbour soon
    \end{itemize}
    
    \textbf{Memorization}\\
    Remember the answer to expensive computations in case it comes up again
    
    \textbf{Amdahl's Law}\\
    If we can increase a fraction $f$ of a task by a factor of $S_{task}$:\\
    $\boxed{S_{total} = \dfrac{1}{(1-f) + \dfrac{f}{S_{task}}}}$
    
    \textbf{Performace measurement}\\
    \textit{Latency}: time to perform a fixed task\\
    \textit{Throughput}: number of tasks in a fixed time\\
    $\implies$ not most contradictory
    
    \textbf{Perf. improvement}\\
    Proc. A is $X$ times faster than . proc. B if:\\
    TODO
    
    \textbf{Averaging perf.}\\
    Latencies can be added, but not throughputs:\\
    $Lat(P1 + P2) = Lat(P1) + Lat(P2)$\\
    $Thr(P1 + P2) = \frac{1}{\frac{1}{Thr(P1)} + \frac{1}{Thr(P2)}}$
    
    \textbf{Averages}\\
    Arithmetic (for time): $\frac{1}{N}\sum_i f(i)$\\
    Harmonic (for rates): $\frac{N}{\sum_i \frac{1}{f(i)}}$\\
    Geometric (for ratio): $\sqrt[N]{\prod_i f(i)}$
    
    \textbf{Processor perf.}\\
    $Perf = \underbrace{\frac{instructions}{program}}_{executed\ instructions} \times \underbrace{\frac{cylces}{instruction}}_{CPI} \times \underbrace{\frac{time}{cycle}}_{clock}$ 
    

\end{minipage}
}\fbox{
\begin{minipage}[t][\boxheight][c]{\boxwidth}

    \textbf{MIPS}\\
    $MIPS= CPI \times clock \times 10^{-6}$
    
    \textbf{Power-perf. metrics}\\
    Power-delay: $PDP = P_{avg} \times t$\\
    Energy-delay: $EDP = PDP \times t$\\
    Energy-delay squared: $EDDP = EDP * t$
    
    \textbf{ISA}\\
    Instruction Set Architecture: abstraction layer between the software and hardware (facilitates dev., provides portability) 
    
    \textbf{Good ISA traits}
    \begin{itemize}
        \item Programmability: easy to express programs efficiently
        \item Implementability: easy to make high-perf / low-power / low-cost / etc. implementations
        \item Compatibility: new generations do not break old software
    \end{itemize}
    
    \textbf{Sequential Model}
    \begin{enumerate}
        \item Fetch
        \item Decode
        \item Read
        \item Execute
        \item Write
    \end{enumerate}
    
    \textbf{Seven dimensions of ISA}
    \begin{enumerate}
        \item Class: load-store, register-memory,\dots
        \item Instr. encoding: fixed or variable length
        \item Types \& sizes of operands: 8-bit, 16-bit,\dots
        \item Addressing mode: register, immediate, \dots
        \item Memory addressing: byte-aligned, not aligned
        \item Operations: arithmetic, control, \dots
        \item Control flow: conditional branches, jump, \dots
    \end{enumerate}
    
    \textbf{Class: operand model}\\
    How many explicit operands?
    \begin{itemize}
        \item 3: general purpose
        \item 2: multiple explicit accumulators
        \item 1: one implicit accumulator
        \item 0: hardware stack
        \item 4+: special cases
    \end{itemize}
    
    \textbf{Instruction length}
    \begin{itemize}
        \item Fixed length: simple implementation, low code density
        \item Variable length: high code density, complex fetch
    \end{itemize}
    

    
\end{minipage}
}\fbox{
\begin{minipage}[t][\boxheight][c]{\boxwidth}


	TODO



\end{minipage}
}

\fbox{
\begin{minipage}[t][\boxheight][c]{\boxwidth}

    TODO
    
\end{minipage}
}\fbox{
\begin{minipage}[t][\boxheight][c]{\boxwidth}

    TODO
    
\end{minipage}
}\fbox{
\begin{minipage}[t][\boxheight][c]{\boxwidth}


	TODO



\end{minipage}
}
\end{small}
\end{document}
