\documentclass[10pt,letterpaper,landscape]{report}

\usepackage[utf8]{inputenc}
\usepackage[english]{babel}
\usepackage[T1]{fontenc}
\usepackage{lmodern}


\usepackage{amsmath}
\usepackage{amsfonts}
\usepackage{amssymb}

\usepackage{enumitem}
\setlist{nosep}


\usepackage{graphicx}

\usepackage[left=-0.5cm,right=-0.5cm,top=-0.3cm,bottom=-0.3cm]{geometry}


\author{Adrien Cosson}
\title{Fiches de révision}

\newcommand{\boxheight}{21.59cm}
\newcommand{\boxwidth}{8.85cm}


\begin{document}
\begin{small}
\fbox{
\begin{minipage}[t][\boxheight][c]{\boxwidth}
    \textbf{STEP 1}\\
    \textbf{1 - Forward (3D scene pt $\rightarrow$ 2D px)}\\
    World point $P_{w} = [x_{w} y_{w} z_{w}]^{T}$ \\
    turns into camera point $P_{c} = [x_{c} y_{c} z_{c}]^{T}$

    \textbf{2 - Forward subroutine}\\
    Camera pose: $\left[ \underbrace{R}_{3\times3} \vert \underbrace{\tau}_{3\times1} \right]$
    
    $P_c = R . P_w + \tau$
    
    \textbf{3 - Backprojection (2D $\rightarrow$ 3D)} \\
    $P_w = R^T (P_c - \tau) = R^T P_c - R^T \tau$\\
    
    \textbf{STEP 2}\\
    \textbf{1 - Perspective projection $P_c \rightarrow [u;v;1]^T$}
    
    \textbf{2 - Get coordinates}\\
    $u = x_c / z_c \qquad v = y_c / z_c $
    
    \textbf{3 - Turn pinhole 2D point $[u;v]$ into 3D}\\
    $[u;v] \rightarrow [u;v;1]$\\
    
    \textbf{STEP 3}\\
    \textbf{1 - Apply nonlinear lens distortion}\\
    $[u;v] \rightarrow (x,y)$ ("film plane coordinates")
    
    \textbf{2 - 1 parameter radial lens distortion}\\
    $x = u(1 + K_1(u^2 + v^2)) \qquad y = v(1 + K_1(u^2 + v^2))$
    
    \textbf{3 - Invert this ($(x,y) \rightarrow [u;v]$)}\\
    Compute iteratively with $r_{temp} = x^2 + y^2$\\
    Then $\left\{
    \begin{matrix} 
    \tilde{u} = x / ( 1 + K_1 r_{temp}) \\ 
    \tilde{v} = y / ( 1 + K_1 r_{temp}) 
    \end{matrix} 
    \right. $ \\
    and then start again with $r_{temp} = \tilde{u}^2 + \tilde{v}^2$\\
    
    \textbf{STEP 4}\\
    \textbf{1 - "Film coordinates" $\rightarrow$ pixel $(P_x, P_y)$} \\
    \textbf{2 - Coordinates of the pixel}\\
    $\begin{bmatrix}
    P_x \\ P_y \\ 1
    \end{bmatrix} = 
    \begin{bmatrix}
    f & \alpha & c_x \\
    0 & f & c_y \\
    0 & 0 & c_y
    \end{bmatrix}
    \times 
    \begin{bmatrix}
    x \\ y \\ z
    \end{bmatrix}$
    
    Hence 
    $\left\{
    \begin{matrix}
    P_x = fx- c_x \\
    P_y = fy - c_y
    \end{matrix}
    \right.
    $
    
    \textbf{3 - Get film coordinates from pixel coordinates} \\
    $\left\{
    \begin{array}{l}
    x = \dfrac{P_x - c_x}{f} \\
    y = \dfrac{P_y - c_y}{f}
    \end{array}
    \right.
    $\\
    
    \textbf{Harris Corner Detection (strong gradient)} \\
    Maximize $E(u,v) = \sum_{x,y} \left[I(x+u, y+v) - I(x, y)\right]^2$ where $(x,y)$ and $(u,v)$ are patches coordinates
    
    \textbf{Harris Corner Derivation}\\
    Take a first order Taylor estimation of the previous equation: \\
    $\begin{bmatrix} u \\ v
    \end{bmatrix}FILL\ THIS\ OUT$
    
    $\mathbf{M} = \left(\sum \begin{bmatrix}I_x^2 & I_x I_y \\ I_x I_y & I_y^2\end{bmatrix}\right)$ where $I_\alpha$ is the derivative of $I$ in regard to the variable $\alpha$
    
    
\end{minipage}
}\fbox{
\begin{minipage}[t][\boxheight][c]{\boxwidth}

    \textbf{Classifying Patch types - Harris method} \\
    $\lambda_1$ and $\lambda_2$ are the eigenvalues of $\mathbf{M}$.
    \begin{itemize}
        \item both are small: flat surface
        \item only one of them is small: edge
        \item both are large: corner
    \end{itemize}
    
    $\left\{\begin{array}{l}det(\mathbf{M}) = \lambda_1 \lambda_2 \\
    trace(\mathbf{M}) = \lambda_1 + \lambda_2\end{array}\right.$
    
    If $R = det(\mathbf{M}) - k(trace(\mathbf{M})^2$ (Harris R-score) is:
    \begin{itemize}
        \item Large negative: edge
        \item Around 0: flat
        \item Large positive: corner
    \end{itemize}
    
    \textbf{Hessian Matrix (high principal curvature)} \\
    $S(x,y) = S(0,0) + \frac{1}{2}\begin{bmatrix} x & y\end{bmatrix}
    \begin{bmatrix} \frac{\partial^2S}{\partial x^2} & \frac{\partial^2 S}{\partial x \partial y} \\ 
    \frac{\partial^2 S}{\partial x \partial y} & \frac{\partial^2S}{\partial y^2} 
    \end{bmatrix}
    \begin{bmatrix} x \\ y\end{bmatrix}$
    \begin{itemize}
        \item     Search locations where $det(\mathbf{H})=\lambda_1\lambda_2$ is large and positive (meaning a high mountain or valley in intensity), specifically larger than its 8 neighbours
        \item Search places where the absolute of the Laplacian is large.     Laplacian: $tr(\mathbf{H}) = \nabla^2I(x,y) = I_{xx}(x,y) + I_{yy}(x,y) $
        \item     Combine both: search places where: $\frac{tr(\mathbf{H})^2}{det(\mathbf{H})} , \frac{(r+1)^2}{r}$ with $r \in [8;10]$
    \end{itemize}
    
    \textbf{Transformation invariance}\\
    Both methods are resistant to translation (doesn't change intensity) and rotation (doesn't change eigenvalues)\\
    No resistance to scaling for a constant patch size, yes otherwise
    
    \textbf{Blob detection in 2D}\\
    Define the \textit{characteristic scale} as the scale at which the normalized Laplacian of a Gaussian gives the biggest response. This characteristic scale is a good value for a patch size
    
    \textbf{Intensity normalization}\\
    Normalize patches before trying to match them: $\hat{f} = \dfrac{f - \Bar{f}}{\sqrt{\sum \left(f - \Bar{f}\right)^2}}$
    
    \textbf{Normalized correlation}\\
    Use NC to  match patches: $NCC(f,g) = \sum_{\{i,j\}\in\mathbb{R}} \hat{f}(i,j)\hat{g}(i,j)$
    
    \textbf{SIFT steps}
    \begin{enumerate}
        \item Find interest patches as described before
        \item Find patch characteristic orientation: form histogram of gradients orientation in the patch
    \end{enumerate}
    $\Rightarrow$ each key point has a center point (location), orientation (rotation) and radius (scale)
    
\end{minipage}
}\fbox{
\begin{minipage}[t][\boxheight][c]{\boxwidth}


	TODO



\end{minipage}
}

\fbox{
\begin{minipage}[t][\boxheight][c]{\boxwidth}

    TODO
    
\end{minipage}
}\fbox{
\begin{minipage}[t][\boxheight][c]{\boxwidth}

    TODO
    
\end{minipage}
}\fbox{
\begin{minipage}[t][\boxheight][c]{\boxwidth}


	TODO



\end{minipage}
}
\end{small}
\end{document}
