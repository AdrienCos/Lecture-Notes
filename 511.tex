\documentclass[10pt,letterpaper,landscape]{report}

\usepackage[utf8]{inputenc}
\usepackage[english]{babel}
\usepackage[T1]{fontenc}
\usepackage{lmodern}


\usepackage{amsmath}
\usepackage{amsfonts}
\usepackage{amssymb}

\usepackage{enumitem}
\setlist{nosep}


\usepackage{graphicx}

\usepackage[left=-0.5cm,right=-0.5cm,top=-0.3cm,bottom=-0.3cm]{geometry}


\author{Adrien Cosson}
\title{Fiches de révision}

\newcommand{\boxheight}{21.59cm}
\newcommand{\boxwidth}{8.85cm}


\begin{document}
\begin{small}
\fbox{
\begin{minipage}[t][\boxheight][c]{\boxwidth}

    \textbf{Contents of a process}\\
    Stored in a PCB:
    \begin{itemize}
        \item Memory (code, data, stack, heap)
        \item Registers/SP/PC
        \item Page table
        \item OS information (files, sockets, accounting,...)
    \end{itemize}
    
    \textbf{Context switch}
    \begin{enumerate}
        \item Push registers
        \item Save stack pointer
        \item Save pointers to address space
        \item Find next process to run, and its PCB
        \item Load pointers to address space
        \item Load stack pointer
        \item Pop registers
        \item Return
    \end{enumerate}
    
    \textbf{User vs kernel level threads}
    \begin{itemize}
        \item User-level switches more efficiently
        \item Kernel-level does better resource allocation
    \end{itemize}
    
    \textbf{Synchronization methods}
    \begin{itemize}
        \item Blocking
        \item Busy waiting
    \end{itemize}
    
    \textbf{Problem with critical sections}\\
        A process halting in a CS block all others that need to access it
        
    \textbf{System latency}\\
    Largest number of steps the system can take without any of the processes successfully completing an operation in a non-blocking implementation
    
    \textbf{Process latency}\\
    Same but for wait-free implementation
    
    \textbf{Optimistic concurrency control}\\
    Assume that no one else is modifying the same data as you, perform all your operations on a shadow copy of the data structure, and commit it once you are finished. If the commit fails, restart.
    
    \textbf{Event ordering approaches}
    \begin{itemize}
        \item Causality: if event X happens, then event Y happens
        \item Time-based (use a clock to get a total ordering)
    \end{itemize}

    \textbf{Partial ordering}\\
    We define the partial order $\rightarrow$ as:
    \begin{itemize}
        \item If $a$ and $b$ are events, and $a$ happens before $b$, then $a\rightarrow b$
        \item If $a$ is a send, and $b$ its receive, then $a\rightarrow b$
        \item $a\rightarrow b, b\rightarrow c \implies a\rightarrow c$
    \end{itemize}
    
    \textbf{Logical clock}
    \begin{itemize}
        \item Each process $P_i$ gets a clock $C_i$
        \item $C_i(a)$ is the time at which event $a$ happened for process $P_i$
        \item $a\rightarrow b \implies C_i(a) < C_i(b)$
        \item If $a$ is send by $P_i$ and $b$ is receive by $P_j$, then\\ $C_j(b) = \max(C_j, C_i(a)) + 1)$
    \end{itemize}
    
    \textbf{Getting a total order}\\
    Sort all events based on their logical clock values. Break ties with a deterministic rule.

\end{minipage}
}\fbox{
\begin{minipage}[t][\boxheight][c]{\boxwidth}


	\textbf{Atomic Operations}
	\begin{itemize}
	    \item \texttt{test\&set(x)}: sets \texttt{temp} to \texttt{x}, \texttt{x} to \texttt{BUSY} and returns \texttt{temp}
	    \item \texttt{fetch\&add(x, y)}: sets \texttt{temp} to \texttt{x}, \texttt{x} to \texttt{x+y} and returns \texttt{temp} 
	    \item \texttt{compare\&swap(A,B,C)}: if \texttt{A==C}, swaps \texttt{A} and \texttt{B}
	\end{itemize}

    \textbf{Concurrent Data Structures goals}
    \begin{itemize}
        \item Non-blocking: some process will complete in a finite number of steps
        \item Wait-free: all processes will complete in a finite number of steps
    \end{itemize}
    
    \textbf{Physical clock conditions}
    \begin{itemize}
        \item $\left| \frac{dC_i}{dt} -1 \right| < k \ll 1$
        \item $|C_j(t) - C_i(t) | < e$
        \item $C_i(t + \mu) - C_i(t) > \mu (1-k)$
        \item For $C_j(t + \mu) - C_i(t) > 0, \mu > \dfrac{e}{1-k}$ 
    \end{itemize}
    
    \textbf{Physical clock receive}\\
    $C_j(t') = \max(C_j(t'), C_i(t) + \mu)$

    \textbf{Lamport Mutex Algo}
    \begin{itemize}
        \item A process sends a REQUEST to all processes with a timestamp
        \item All processes place the request in a local queue, and respond with a ACK + timestamp
        \item A process sends a timestamped RELEASE to all others
        \item All processes remove the request from the local queue
        \item A process gets the resource if it has the most recent request and everyone has ACKed him
    \end{itemize}
    $\implies O(3(N-1))$ messages complexity
    
    \textbf{Ricart \& Agrawala Algo}\\
    Only send an ACK when you think the request should proceed before you.\\
    $\implies O(2(N-1))$
    
    \textbf{Suzuki \& Kasami Algo.}\\
    One process has PRIVILEGE at all time, and only passes it along if it does not need it anymore, upon request.\\
    $\implies O(N)$ messages

    
    \textbf{Remote Procedure Call Goals}
    \begin{itemize}
        \item Easy to use (close to normal procedure calls)
        \item Efficient communication (close to HW transmission time)
    \end{itemize}
    
    \textbf{RPC components}
    \begin{itemize}
        \item Runtime: hides the details of the transport mechanism
        \item Stub: hides the difference between local and remote call
    \end{itemize}
    
    \textbf{RPC architecture}\\
    Process and stub share a single address space. Each runtime can handle multiple client/server at one
    
    \textbf{RPC stub generation}\\
    User defines an interface definition, and the RPC compiler generates cstubs and sstubs
    
    \textbf{RPC binding}\\
    Client specifies a service name. The mapping from service to machine ID is stored in a name server. 
    
 
    
    
\end{minipage}
}\fbox{
\begin{minipage}[t][\boxheight][c]{\boxwidth}

    
    \textbf{Name server population}
    \begin{itemize}
        \item Server calls \texttt{export()} in sstub
        \item sstub calls \texttt{ExportInterface()} in srt
        \item \texttt{ExportInterface()} sends a message to Grapevine and registers the server
        \item srt maintains data about the export in a table (contains if, dispatcher, machine ID)
    \end{itemize}
    
    \textbf{Name server use}
    \begin{itemize}
        \item Client calls \texttt{import()} in cstub
        \item cstub calls \texttt{ImportInterface()} in crt
        \item crt sends a message to Grapevine
        \item Grapevine sends address of exporter back
        \item crt sends message to srt for binding information
        \item cstub stores binding information (netwk addr, ID, table idx) 
    \end{itemize}
    
    \textbf{RPC network protocol optimization}\\
    Use server return value as call ACK, and client next call as return ACK. If needed, send an actual ACK while work is processed
    
    \textbf{Call ID purposes}
    \begin{itemize}
        \item Multiplex result back to caller at crt
        \item Track answered calls at srt
        \item crt can be stateless
        \item System clock is enough to generate call ID
    \end{itemize}
    
    \textbf{sRPC security guarantees}
    \begin{itemize}
        \item Guarantee to caller that only callee can perform the work
        \item Allow callee to verify the ID of caller
        \item Prevent eavesdropping
        \item Prevent tampering
        \item Prevent replay attacks
    \end{itemize}
    
    \textbf{sRPC authentication}\\
    Kerberos, ticket = $\{\{CK, T, A\}^{KB}, X, B, CK\}^{KA}$, with:
    \begin{itemize}
        \item $CK$: channel key
        \item $T$: nonce for B
        \item $X$: nonce from A
    \end{itemize}
    
    \textbf{sRPC authentication}\\
    A calls B. B responds with an RFA. A responds with its identity and the Kerberos ticket. B responds with the workload response.
    
\end{minipage}
}

\fbox{
\begin{minipage}[t][\boxheight][c]{\boxwidth}

    
    \textbf{Steps to build a consistent DSM}
    \begin{itemize}
        \item Write propagation
        \begin{itemize}
            \item Find out who has copies
            \item Make sure they see the latest value
        \end{itemize}
        \item Read/Write atomicity
        \begin{itemize}
            \item Make sure operations are done one after another in some serial order
        \end{itemize}
    \end{itemize}
    
    \textbf{Coherence protocol types}
    \begin{itemize}
        \item Invalidation-based (writing invalidates other copies)\\
        $\implies$ better when sharing behaviour changes (often the case)
        \item Update-based (writing [propagates value]\\
        $\implies$ better when sharing behaviour is constant
    \end{itemize}
    
    \textbf{Directory}\\
    Data structure where each row is a shared memory location, and has a presence bit for each processor, a status bit (read-only or read-write), as well a lock
    
    \textbf{Where to store a directory?}
    \begin{itemize}
        \item Centralized node (not scalable)
        \item Split the directory by rows and distribute it to all the nodes
        \item Distribute the entries and allow them to move between nodes (complex)
    \end{itemize}
    
    \textbf{Write serialization}\\
    When a write is happening at a location, no other read/write can happen concurrently at the same location
    
    \textbf{Write atomicity}\\
    Write is not visible at any location until it is propagated everywhere
    
    \textbf{File usage characteristics}
    \begin{itemize}
        \item File size: most files are under 10KB (Zipf's law)\\
        $\implies$ set transfer units accordingly
        \item Lifetime: most files are short-lives\\
        $\implies$ don't propagate writes immediately
        \item Sharing: only a few files are actively shared\\
        $\implies$ good case for client-side caching
    \end{itemize}
    
    \textbf{Distributed file systems optimizations}
    \begin{itemize}
        \item System binaries: rarely change, widely replicate
        \item Temp. files: short, unshared, short-lived, kept locally
        \item Mailboxes: frequent updates, rarely cached
    \end{itemize}
    
    \textbf{File caching}\\
    Fully associative cache. Hash the fd+offset, find its position in a hash table of linked list, each entry correspond to an entry in the cache. Only lock for access to the hash table entry, not the entire hash table
    
    \textbf{Distributed write-back cache}\\
    We need a flusher thread that flushes dirty blocks at regular intervals. Requires a dirty block table that points to the cache entries\\
    We also need a harvester that evicts used blocks to maintain a certain hysteresis threshold, with an LRU policy
    
    
\end{minipage}
}\fbox{
\begin{minipage}[t][\boxheight][c]{\boxwidth}

    TODO
    
\end{minipage}
}\fbox{
\begin{minipage}[t][\boxheight][c]{\boxwidth}


	TODO



\end{minipage}
}
\end{small}
\end{document}
